\documentclass[12pt,a4paper]{article}

\usepackage{epcc}
\usepackage{graphics}

\usepackage{verbatim}
\usepackage{listings}
\lstset{language=C++}
\lstset{basicstyle=\ttfamily\footnotesize,frame=single}

\lstnewenvironment{code}[1][]%
{
    \noindent
    \minipage{\linewidth} 
    \vspace{0.5\baselineskip}
    \lstset{#1}
}
{\endminipage}

\lstnewenvironment{longcode}[1][]%
{
    \noindent
    \vspace{0.5\baselineskip}
    \lstset{#1}
}
{}

\newcommand{\term}[1]{\texttt{#1}}


\begin{document}

\title{
    Simulations of the Effect of Blue Nose Disease on a Population of Frogs\\
    \Large{An Actor Model Implementation}
}
\author{b5fe-1fde7d}
\date{\today}

\makeEPCCtitle

\thispagestyle{empty}

\newpage

\pagenumbering{roman}

\tableofcontents

\newpage
\pagenumbering{arabic}


\section{Introduction}
A model has been devised by a group of biologists to simulate the
spread of "Blue Nose Disease" through a population of frogs.
A parallel simulation of this model was implemented
using the "actor model."
This is done by first implementing a parallel actor model framework
using C\verb!++! and MPI, and then by writing the simulation
using this framework.

In writing the actor model framework, much of the design was informed
by considering the population simulation that was to be built with it.
Indeed, as the simulation was being written, some interfaces
and implementation details of the actor framework were reconsidered
and rewritten to allow programs built using it to be
more robust and more easily implemented.
However, at all times, care was taken to ensure that the actor framework
remained as generally applicable as possible.

In implementing both the framework and the simulation,
a test driven development approach was taken to help with ensuring
the correctness of code and to aid in later debugging and refactoring.
However, as parallel programming is hard and in particular,
asynchronous programming, it is difficult to ensure that the tests
written are actually correct.
Further difficulty is derived from the fact that some aspects of the code
can only be tested indirectly, as a programmer may lose access to certain
details they would like to test.
An example is testing whether an actor is, indeed, capable of giving
birth to actors, and that those actors continue to operate
and give birth after the original parent is dead.
This can be tested by getting these children to message some test
class inheriting \term{Actor} and
by measuring increasing loads on the \term{Director},
which will be introduced later as the user interface for
configuring and launching actor model instances.
However, this is only indirect evidence that the expected behaviour
is being carried out, and there may be many bugs hiding in the gaps
in our observations.

\newpage
\section{Actor Model Implementation}
The goal for our implementation of the actor model is to make the
definition of new actors as simple as possible for users,
and to make the launching of parallel actor model jobs as
straightforward as possible.
We aim to allow the user to specify the number of processes
they would like to run the model on and balance the running
of any number of actor across these processes.

To define a type of actor for use in the model, a user
will inherit from the \term{Actor} class.
They may then define some initialization procedures by
overloading the class constructor,
a main function to be run in a loop by the \term{Director} by
overloading the \term{main} function
and some clean up procedures by
overloading the class destructor.
The \term{Actor} class provides interfaces to the inheriting class
to perform necessary actions such as sending and receiving messages,
giving birth to new actors
and dying.

To run an instance of an actor model program,
a user creates an instance of a \term{Director}.
The user then registers the types of actors that can be born in
the simulation with the \term{Director}.
The user may then add an initial actor to the simulation,
receiving a pointer to that actor,
and may then use this actor to create and initialize
new actors in the system
and also retreive information from the system.

The framework makes extensive use of MPI\_Bsend.
As such, a buffer must be allocated and provided to MPI.
This has been left to the user to do.
It would be trivial to abstract the MPI library away completely,
but this was not done here.
It is also required of the user to ensure MPI\_Init and MPI\_Finalize
are called in the appropriate places, and importantly, that the
destructor of any directors be called before MPI\_Finalize is
called.


\subsection{Distributed Factories}
In our design, we require that many classes inheriting from the
\term{Actor} base class may be initialized using a request sent as an MPI
message,
and in particular, that one process may specify the particular
subclass they wish to be initialized and on which process.

To achieve this, we create a general \term{Factory} class which can be used
to enumerate subclasses by registering them with the factory.
A process can then request a subclass be initialized by requesting
it by number.
This, however, requires that the subclasses be registered in the same
order on every process.

The \term{Factory} class is then extended to a \term{DistributedFactory}
class, which allows processes to send requests that a particular
subclass be initialized on a particular process.
A process will receive this request by first querying whether there
are any outstanding requests, and then by generating the
requested actor if there are.
To connect instances of the \term{DistributedFactory} together,
a collective routine is used to create a new communicator,
so connected distributed factories must be initialized simultaneously
on the required processes.

When an instance of a subclass is initialized,
we assume it remains on the same process.
This allows us to simplify the management of running actors
and their communications.
It also allows us to create a global id for that actor,
providing a unique identifier and specifying the
process on which they live.
This global id can then be used to easily target specific actors
when sending messages.
This assumption also allows us to create this global id on the
requesting process.
The id can be returned immediately from the request function
so the requester can immediately identify the child class.

\begin{code}[caption=DistributedFactory Example]
DistributedFactory<BaseClass> df;
df.register_child<SubClass>();

int rank;
MPI_Comm_rank(MPI_COMM_WORLD, &rank);

if(rank == 0) {
    Id child_id = df.request_distributed_child<SubClass>();
    std::cout << "Requested Id: " << child_id.id << std::endl;
}

// Ensure the request has gone out
MPI_Barrier(comm);

if(df.is_child_waiting()) {
    // Not necessarily rank 0!

    DistributedFactory<BaseClass>::Child child =
        df.generate_requested_child();

    // child.id should contain the same value as child_id above
    std::cout << "Retreived Id: " << child.id << std::endl;


    /**
     * base should be a pointer to an instance of SubClass,
     * casted to BaseClass.
     * If do_something is an overloaded virtual function,
     * this should execute SubClass::do_something().
     */
    BaseClass *base = child.child;
    base->do_something();
}
\end{code}


\subsection{Actors}
As mentioned, the \term{Actor} class is the base class
to be inherited by all user defined actors.
To define a new \term{Actor}, the child class must overload the
\term{main} function.
This is function will be executed repeatedly and should be used
to define the behavior of the actor.
Actors may also overload the constructors and destructors to manage
their data.
However, these shouldn't be used to message other actors
or to spawn new actors.
New actors are free to use member variables to stora data or state
and add member functions to their class, which may be accessed by users
in the case an actor is accessible.

The \term{Actor} class comes with several useful functions.
These include functions to message other actors and receive messages,
give birth to other actors, and to die.

A \term{Message} class is used to send and receive messages.
This class also simplifies the management of message memory for scalar
data types, but also allows for arrays to be sent as data.
An actor can send a message to any actor it knows the id of.
A message can be composed of some message data and a message tag.
This is different from the usual MPI tag, as the MPI tag is actually
used to identify actors, so this tag is sent ahead in a separate message
ad metadata.
When sending and accessing data, templating is used to simplify the process.
Data is sent and received between actors as an MPI\_BYTE type.
As such, this may not be portable to heterogenous systems.
It is conceivable that a sort of factory scheme could be used to
send and receive MPI messages with an appropriate MPI datatype.
This, however, was not implemented, and the more straightworward solution
was chosen as the target architecture was not heterogenous.

\begin{longcode}[caption=Example Actor]
class SubActorManager: public Actor {
    SubActorManager():
        _sub_actor_list(NULL),
        _sub_actor_list_size(0)
    {}

    ~SubActorManager() {
        delete[] _sub_actor_list;
    }

    // Some easily identifiable message tags
    enum {
        CREATE_SUB_ACTOR_LIST,
        GET_SUB_ACTOR_LIST,
        DIE
    }

    // The main routine to be executed repeatedly
    void main(void) {
        Message message;

        /**
         * get_message(Message *message) will return true if
         * there is a message waiting and set message
         * appropriately.
         */
        if(get_message(&message)) {

            // Message dependent actions.
            switch(message.tag()) {
                case CREATE_SUB_ACTOR_LIST: {

                    // message.data<int>() will return the data
                    // the message has received casted to an int.
                    _sub_actor_list_size = message.data<int>();

                    _sub_actor_list = new Id[_sub_actor_list_size];
                    for(int i=0; i<_sub_actor_list_size; i++) {

                        // give_birth<SubActor>() will request
                        // a child of type SubActor is born
                        // on some process.
                        // The Id of this child is immediately
                        // returned.
                        _sub_actor_list[i] = give_birth<SubActor>();

                        SubActor::InitData init_data = ...

                        // send_message<type>(id, type data, tag)
                        // will send the data in data to actor id
                        // with the tag tag.
                        send_message<SubActor::InitData>(
                            _sub_actor_list[i],
                            init_data,
                            SubActor::INIT
                        );
                    }
                } break;

                case GET_SUB_ACTOR_LIST: {
                    Id reply_id = message.data<Id>();

                    // If a length parameter is specified after
                    // the data is passed in, it is expected
                    // that a pointer to the data has been passed
                    // in and the number of elements specified
                    // exist.
                    send_message<Id>(
                        reply_id,
                        _sub_actor_list, _sub_actor_list_size,
                        GET_SUB_ACTOR_LIST
                    );
                } break;

                case DIE: {

                    // An example of a manager requesting that
                    // a list of actors under its management
                    // also die when it does.
                    for(int i=0; i<_sub_actor_list_size; i++) {
                        bool ignore=true;
                        send_message<bool>(
                            _sub_actor_list[i],
                            ignore,
                            SubActor::DIE
                        );
                    }

                    die();
                }

            }
        }

    }

private:
     Id *_sub_actor_list;
     int _sub_actor_list_size;
};
\end{longcode}


\subsection{Directors}
The \term{Director} class is the interface a user uses to set up
and execute an actor model job.
The user specifies which \term{Actor} subclasses can be born in
the model through the director.
The user may then add some initial actors to the job and the director
will return pointers to these actors, which the user may then use
to specify some initial state for that actor before running the job.
After, or indeed during, the running of the job, the user may extract
information from the system through these initial actors.
This may be done by other actors in the job sending messages to
these initial actors which the user may then interact with directly.

The director manages the initialization of a distributed factory
and the creation of some communicators for the director and the actors
to communicate over.
It then manages the initialization and execution of actors either
explicitly requested by the user, or requested through the distributed
factory.
The initialization of an actor includes wiring it up with an id
so it can send and receive messages with other actors,
a communicator to send those mesages over
and a pointer to a distributed factory so it can request that other
actors are born.

When an actor is created by a director, it is added to the end of a queue.
Each actor in the queue is, in turn, popped off the queue, its main
function executed, and if it hasn't died since the execution of
that main function, it is added to the end of the queue to be executed
again later.
For actors created through the distributed factory, the director
manages the deallocation of these actors after they die.
For actors requested explicitly be the user, as opposed to those
born of another actor, it is up to the user
to handle their destruction, which should not be performed before the actor
has died.

There are 3 ways for a director to stop running an actor model job.
The first is if a maximum number of \term{ticks} is specified.
A \term{tick} is defined as the execution cycle of one \term{Actor}.
The second is if every director no longer has any live actors to
execute.
The third is if a process explicitly requests the director ends
its run.
This final method is useful for situations where a particular trigger
should be used to terminate the run, rather than relying on a fixed
number of \term{tick}s or on all the actors dying.


\subsection{Performance Considerations}
Directors will compare their list sizes after a given number of \term{tick}s
to determine when every director's list has been emptied.
This routine requires a global barrier, so for simulations where it
is not important to end immediately after there is no work left to do,
this interval can be set to a very long value.
It may also be possible to remove this synchronization altogether
if remote memory accesses are used to determine the loads of different
directors, or if the onus were to be placed entirely on the user
to determine when the director run ends.
While no performance measurements have been made to date,
the framework has been tested to work on up to 64 cores on Morar.


\section{Population Model Simulation}
The goals of the simulation were to implement the model described by
the biologists in such a way that the starting parameters could
be easily changed, and that the system would output the current state
of the cells at the end of the year, and the frog count at some interval.
In this implementation, a year is determined by a wall clock time.
As such, at the end of the year, the frog population and the state
of the cells are output simultaneously.
There are many methods of determining a year, each with their own
merits and drawbacks.
The issue with wall clock time is that the average number of hops
per frog per year can vary wildly with the number of frogs in the
simulation and the number of cores used.
A better metric to use here may have been average hops per frog in
the simulation.
This could be used by polling one or a small number of cells for their
population data, comparing that to the number of cells in the grid and
the number of frogs in the simulation and determining the average
hops per frog from there.

\subsection{Cell Actors}
Cells contain some historical data about the frogs that land on them.
When a frog lands on a cell, it informs the cell that it has landed
on it and of its infection status.
The cell then updates its \term{populationInflux} and
\term{infection\_level} accordingly.

An actor can request that a cell sends its \term{populationInflux} and
\term{infection\_level} to it.
This is the suggested method for monitoring these values in a cell.
It may be tempting for a user to simply make an explicit request
to the director for a number of cells and use them for the simulation.
As they would be on the same process, this would enduce a high
load on that process with respect to other actors messaging the
cell.

A cell can also be instructed to set its \term{populationInflux} and
\term{infection\_level} values to a given value.
This allows for both initializing cells to have some given values
for these terms, or to simulate a monsoon by instructing them to set
these values to 0.

\subsection{Frog Actor}
A frog actor requires several values to be initialized before it
begins moving about the environment.
First, it must be informed of its initial coordinates.
Next, it must be given a list of the cells it will be moving about.
Finally, it must be given the id of an actor it will message when
it has finished being initialized, and then again when it dies.
A frog may optionally be told to become infected.

When a frog is first fully initialized, it requests the state of the cell
it is in.
It will then wait until it receives this data before hopping.
When it has hopped, it will increment a counter tracking the number
of hops it has performed and it will
request the state of the cell it has hopped to before hopping again.
When it receives the state data from the cell, it will add the
\term{populationInflux} value to the current running sum of values
it has received, and add the \term{infection\_level} to a circular
buffer containing the last 500 values it has received.

Immediately after hopping, a frog will test whether it should give
birth, become infected or die, according to the biologists' model
using the functions provided by the biologists.

When testing birth, the frog tests if the current total of hops is
divisible by 300, the number given in the model.
It then divides the currently tracked sum of \term{populationInflux}
values by 300 to find the average, and gives birth if the
function provided by the biologists says it should.
If it gives birth, it will also send the necessary initialization
data to the newly created child.

When testing for infection, the frog finds the average of all the
values in the circular buffer of \term{infection\_level}s and
feeds this into the biologists' function, becoming infected if
the function tells it to.

When testing for death, the frog tests if the current hop count is
divisible by 700.
If so, it runs the biologists' function to test whether or not it should
die, dying if told to do so.

\subsection{Simulation Actor}
The simulation actor is used to create a grid of cells, and to
give birth and initialize the initial batch of frogs.
It is also used to track the state of the grid cells,
the number of frogs in the simulation
and to initiate the monsoon.

The simulation actor is added to the director as a user added actor
on the root process.
A user then calls the initialize method with the parameters of the simulation.
The user then runs the director.
The simulation written can optionally accept command line arguments
reflecting these paramaters.

The initialize method will spawn 16 grid cells.
A fixed number of cells was chosen, because the function provided
by the biologists is hard coded to enumerate 16 cells.
The method will then spawns the requested number of frogs,
initializing them by sending the generated grid of cells,
setting their initial positions to $(0,0)$
and setting the actor they should notify of their initialization and
death to the simulator.
It will also set a requested number of frogs to infected.
This initialization is necessary, because if all frogs are healthy
and there is no disease in the environment, no frog will ever
become infected.
This mimics the effect of a year ending with a certain number of infected
frogs and the population data in the cells being reset.

Each time the simulation actor runs, it will check for new messages.
It may receive a message from a frog stating whether it has been initialized
or if it has died.
The simulation actor will adjust a frog counter accordingly.
It may also receive a number of messages containing the current population
data of a cell.
This should only happen soon after it has explicitly requested the data
at the end of a year, so it outputs this data to the screen immediately.

On every run, the actor checks the current wall time.
If a specified number of seconds have elapsed since the last year,
the simulation actor will perform a number of end-of-year tasks.
It will output the current year, the current number of frogs in
the simulation and to request that cells send it their population
data.
As already mentioned, when the simulator receives these messages, it outputs
the results immediately, as we expect the responses to come soon after
this event.
It then requests the cells reset their population data.
The population request and population reset messages are sent
immediately, one after the other.
This assumes that messages are received in order.
As MPI guarantees this, it is a reasonable assumption to make.

If the requested number of years to simulate has passed,
the simulator will ask the director to end the run.
At this time, the simulator will also check if the current number of
frogs in the simulation has passed 100.
If so, it will ask the director to stop and will output an error
stating the number of frogs in the system has passed this number.

The simulator will also output frog populations at a given interval.
The time check and update are performed much like the end of year.


\subsection{Results}
In the simulation written, the survival of the frogs depends strongly
on the length of a year.
It appears that almost all the frogs in the simulation become infected
rather rapidly, even if only a single infected frog is introduced.
As such, the infection level tends to be roughly equal to the
population influx in all cells.

Different cells in the simulation tend to have roughly similar population
data at any one time.
This seems reasonable, as if the frogs are moving randomly, they are
likely to disperse evenly about the system.

The primary factor determining the survival or extinction appears to be
the maximum population influx a cell can attain in a year.
To run more stable simulations, it may be better to tie the length
of a year to the influx per frog of some cell or group of cells.
This could be more accurate if some figures are known about the number
of hops an individual frog is expected to do in a year.
It also circumvents the problem of the simulation runing faster
when run on more cores, requiring the length of a year to be tuned
to a reasonable number.


\section{Conclusion and Discussion}
MPI is quite a good fit for the actor model, as messages can be
targeted at specific actors on a process using message tags.
The guarantee of message ordering from MPI also allows metadata to
be passed along with messages, as was done here, to allow messages
to include a further tag.
While this could have been implemented by folding the message tag
into the MPI tag or the message,
the use of an extra message as a header allows richer metadata to be
passed between actors.

C\verb!++! is also rather well suited, as defining new actors is quite
a clean procedure.
Much of the gritty details of the implementation can be hidden away
from the user, allowing them to simply inherit from an actor class
and use some methods provided by it.
Indeed, if it is later decided to change the message passing library used,
this can be done without affecting much of the user's code.
In this implementation, we require a user to explicitly initialize MPI,
attach a buffer and finalize it.
This could easily be wrapped in some initialization and finalization
functions in the library, or even in the director.

This implementation relies rather heavily on templating to effectively
hide much of the details of the implementation from the user
and to allow objects to be enumerated and identified in MPI messages.
The factory class, for example, uses templating to enumerate and identify
some factory functions for creating new instances of subclasses in a
manner that these identifiers can be safely sent across MPI.
One drawback of relying on templating is that the templated code
must be compiled with every file that might call it with a new template.
This can have a large impact on the time taken to compile the code.
This is a small price to pay for the increased ease of use and
cleanness of templates.

There is also the ugly requirement of requiring the user to register
any actor classes that may be born in the simulation.
Automatic subclass registration was investigated along with using
RTTI and other promising C\verb!++! features.
Due to poor reliability on the part of execution routes of highly
asynchronous algorithms and on the part of compiler implementations,
the safer, if less glamorous route was taken.

The simulation written delivered the interesting result that
under the current model,
the survival or extinction of the population of frogs depends strongly
on the length of a year.
That is, the maximum possible value the populationInflux of a cell can
reach.
It also found that a single infected frog will rapidly infect the
entire population of frogs.

To simulate a stable population on 32 cores as has been done here,
a rather low year time must be used.
With the current implementation, this means that a year may pass
before a cell has had a chence to return its population data and have
it output to the screen.
As such, a little mangling of data may occur.
For similar timing reasons, the first year should be discarded.


\newpage
\appendix
\section{Sample Data}
The following is the output for the parameters
\begin{itemize}
    \item Initial number of frogs: 34
    \item Initial number of infected frogs: 1
    \item Maximum number of allowed frogs: 100
    \item Frog population output frequency: 0.005 seconds
    \item Number of cells: 16
    \item Years to model: 100
    \item Length of year: 0.01 seconds
\end{itemize}

The actual output has more newlines, but the mangling done by LaTeX
has served well to compress the data into a managable length.

{\footnotesize
YEAR: 1
FROG POPULATION: 28
FROG POPULATION: 28
DATA: (1,0,0)
DATA: (3,0,0)
DATA: (5,0,0)
DATA: (7,0,0)
DATA: (8,0,0)
DATA: (11,0,0)
DATA: (12,0,0)
DATA: (13,0,0)
DATA: (14,0,0)
DATA: (15,0,0)
DATA: (10,0,0)
DATA: (0,0,0)
DATA: (9,0,0)
DATA: (6,1,0)
DATA: (2,6,0)
FROG POPULATION: 33
DATA: (4,7,0)

YEAR: 2
FROG POPULATION: 34
FROG POPULATION: 34
DATA: (1,18,0)
DATA: (2,6,0)
DATA: (4,7,0)
DATA: (5,20,1)
DATA: (6,8,2)
DATA: (8,9,0)
DATA: (11,7,1)
DATA: (12,8,0)
DATA: (13,10,0)
DATA: (14,14,1)
DATA: (15,16,0)
DATA: (0,9,0)
DATA: (7,13,0)
DATA: (9,8,0)
DATA: (3,11,0)
FROG POPULATION: 34
DATA: (10,12,0)

YEAR: 3
FROG POPULATION: 34
DATA: (1,20,0)
DATA: (2,11,0)
DATA: (4,17,1)
DATA: (7,8,0)
DATA: (8,10,0)
DATA: (9,11,0)
DATA: (10,13,0)
DATA: (11,7,0)
DATA: (12,11,0)
DATA: (13,11,0)
DATA: (14,10,0)
DATA: (15,7,0)
DATA: (0,12,0)
FROG POPULATION: 34
DATA: (6,0,0)
FROG POPULATION: 34
DATA: (3,12,0)
DATA: (5,23,1)

YEAR: 4
FROG POPULATION: 34
DATA: (1,13,0)
DATA: (2,13,0)
DATA: (3,7,0)
DATA: (5,5,0)
DATA: (7,24,0)
DATA: (8,11,0)
DATA: (9,13,0)
DATA: (11,9,0)
DATA: (12,15,0)
DATA: (13,16,0)
DATA: (14,15,0)
DATA: (15,15,0)
DATA: (0,10,0)
FROG POPULATION: 34
FROG POPULATION: 34
DATA: (4,14,0)
DATA: (6,13,0)
DATA: (10,11,0)

YEAR: 5
FROG POPULATION: 34
DATA: (1,14,0)
DATA: (2,20,1)
DATA: (4,16,0)
DATA: (6,12,0)
DATA: (8,14,0)
DATA: (9,20,0)
DATA: (10,6,0)
DATA: (11,13,0)
DATA: (12,18,1)
DATA: (13,11,0)
DATA: (14,12,0)
DATA: (15,9,0)
DATA: (0,22,0)
FROG POPULATION: 34
FROG POPULATION: 34
DATA: (3,15,0)
DATA: (7,21,1)
DATA: (5,14,0)

YEAR: 6
FROG POPULATION: 34
DATA: (1,13,0)
DATA: (2,12,1)
DATA: (3,7,0)
DATA: (5,7,0)
DATA: (7,9,1)
DATA: (8,10,0)
DATA: (9,10,0)
DATA: (11,7,0)
DATA: (12,12,0)
DATA: (13,12,0)
DATA: (14,18,0)
DATA: (15,13,1)
DATA: (0,11,0)
FROG POPULATION: 34
DATA: (4,3,0)
FROG POPULATION: 34
DATA: (6,6,0)
DATA: (10,21,0)

YEAR: 7
FROG POPULATION: 34
DATA: (1,18,0)
DATA: (4,20,0)
DATA: (6,13,0)
DATA: (8,15,0)
DATA: (9,24,1)
DATA: (10,13,0)
DATA: (11,14,0)
DATA: (12,15,0)
DATA: (13,13,0)
DATA: (14,23,0)
DATA: (15,12,0)
DATA: (0,13,0)
FROG POPULATION: 34
DATA: (3,12,1)
DATA: (7,16,0)
FROG POPULATION: 34
DATA: (5,17,0)
DATA: (2,18,2)

YEAR: 8
FROG POPULATION: 34
DATA: (1,14,0)
DATA: (2,11,0)
DATA: (3,8,0)
DATA: (5,7,0)
DATA: (7,9,0)
DATA: (8,12,0)
DATA: (9,11,0)
DATA: (11,9,0)
DATA: (12,8,0)
DATA: (13,14,0)
DATA: (14,11,0)
DATA: (15,8,1)
DATA: (0,14,0)
FROG POPULATION: 34
DATA: (4,0,0)
DATA: (6,17,0)
FROG POPULATION: 34
DATA: (10,13,0)

YEAR: 9
FROG POPULATION: 34
DATA: (1,16,0)
DATA: (4,18,0)
DATA: (6,11,0)
DATA: (8,12,0)
DATA: (9,15,0)
DATA: (10,7,0)
DATA: (11,13,0)
DATA: (12,13,0)
DATA: (13,18,0)
DATA: (14,18,0)
DATA: (15,7,0)
DATA: (0,10,0)
FROG POPULATION: 34
DATA: (5,11,0)
DATA: (3,0,0)
DATA: (7,19,0)
FROG POPULATION: 34
DATA: (2,15,0)

YEAR: 10
FROG POPULATION: 34
DATA: (1,19,1)
DATA: (2,9,1)
DATA: (3,17,1)
DATA: (5,19,1)
DATA: (7,14,1)
DATA: (8,26,1)
DATA: (9,16,0)
DATA: (10,30,2)
DATA: (11,20,1)
DATA: (12,33,2)
DATA: (13,24,1)
DATA: (14,25,3)
DATA: (15,19,1)
DATA: (0,22,3)
FROG POPULATION: 34
DATA: (4,0,0)
DATA: (6,36,4)
FROG POPULATION: 34

YEAR: 11
FROG POPULATION: 34
DATA: (1,23,0)
DATA: (4,14,2)
DATA: (5,12,0)
DATA: (6,10,0)
DATA: (8,18,0)
DATA: (9,22,0)
DATA: (10,17,1)
DATA: (11,15,0)
DATA: (12,13,0)
DATA: (13,17,0)
DATA: (14,23,1)
DATA: (15,17,0)
DATA: (0,19,0)
DATA: (3,0,0)
FROG POPULATION: 34
DATA: (7,27,0)
FROG POPULATION: 34
DATA: (2,22,0)

YEAR: 12
FROG POPULATION: 34
DATA: (1,12,0)
DATA: (2,7,0)
DATA: (4,35,2)
DATA: (5,19,0)
DATA: (7,25,0)
DATA: (8,17,0)
DATA: (9,13,1)
DATA: (10,30,1)
DATA: (11,18,1)
DATA: (12,25,2)
DATA: (13,21,0)
DATA: (14,23,0)
DATA: (15,15,0)
DATA: (0,19,1)
FROG POPULATION: 34
DATA: (6,21,1)
FROG POPULATION: 34
DATA: (3,27,0)

YEAR: 13
FROG POPULATION: 34
DATA: (1,23,2)
DATA: (3,11,0)
DATA: (5,14,0)
DATA: (7,26,1)
DATA: (8,21,1)
DATA: (9,17,2)
DATA: (10,15,0)
DATA: (11,25,1)
DATA: (14,17,0)
DATA: (15,15,0)
DATA: (0,32,0)
FROG POPULATION: 34
FROG POPULATION: 34
DATA: (4,22,0)
DATA: (2,24,2)
DATA: (6,18,0)

YEAR: 14
FROG POPULATION: 34
DATA: (2,1,0)
DATA: (4,1,0)
DATA: (5,1,0)
DATA: (6,0,0)
DATA: (8,0,0)
DATA: (9,2,0)
DATA: (10,2,0)
DATA: (11,1,0)
DATA: (14,1,0)
DATA: (0,0,0)
FROG POPULATION: 34
FROG POPULATION: 34
DATA: (3,1,0)
DATA: (7,4,0)

YEAR: 15
FROG POPULATION: 34
DATA: (3,0,0)
DATA: (5,0,0)
DATA: (7,0,0)
DATA: (8,0,0)
DATA: (9,1,0)
DATA: (10,0,0)
DATA: (14,0,0)
DATA: (0,1,0)
FROG POPULATION: 34
DATA: (4,1,0)
FROG POPULATION: 34
DATA: (6,0,0)
DATA: (2,1,0)

YEAR: 16
FROG POPULATION: 34
DATA: (2,0,0)
DATA: (4,1,0)
DATA: (5,0,0)
DATA: (6,0,0)
DATA: (8,0,0)
DATA: (9,0,0)
DATA: (10,0,0)
DATA: (14,1,0)
DATA: (0,0,0)
FROG POPULATION: 34
DATA: (3,0,0)
DATA: (7,0,0)
FROG POPULATION: 34

YEAR: 17
FROG POPULATION: 34
DATA: (3,0,0)
DATA: (5,0,0)
DATA: (7,0,0)
DATA: (10,0,0)
DATA: (0,0,0)
FROG POPULATION: 34
DATA: (4,0,0)
DATA: (6,0,0)
FROG POPULATION: 34
DATA: (2,0,0)

YEAR: 18
FROG POPULATION: 34
DATA: (2,0,0)
DATA: (4,0,0)
DATA: (5,0,0)
DATA: (6,0,0)
DATA: (10,0,0)
DATA: (0,0,0)
FROG POPULATION: 34
DATA: (3,0,0)
DATA: (7,0,0)
FROG POPULATION: 34

YEAR: 19
FROG POPULATION: 34
DATA: (3,0,0)
DATA: (5,0,0)
DATA: (7,0,0)
DATA: (10,0,0)
DATA: (0,0,0)
FROG POPULATION: 34
DATA: (4,0,0)
DATA: (6,0,0)
FROG POPULATION: 34
DATA: (2,0,0)

YEAR: 20
FROG POPULATION: 34
DATA: (2,0,0)
DATA: (4,0,0)
DATA: (5,0,0)
DATA: (6,0,0)
DATA: (10,0,0)
DATA: (0,0,0)
DATA: (3,0,0)
FROG POPULATION: 34
DATA: (7,0,0)
FROG POPULATION: 34

YEAR: 21
FROG POPULATION: 34
DATA: (4,0,0)
DATA: (7,0,0)
DATA: (10,0,0)
DATA: (0,0,0)
DATA: (6,0,0)
FROG POPULATION: 34
DATA: (2,0,0)
DATA: (3,0,0)
DATA: (5,0,0)
DATA: (9,0,0)
DATA: (9,0,0)
DATA: (11,2,0)
DATA: (12,28,1)
DATA: (14,0,0)
DATA: (15,2,0)
DATA: (1,3,0)
DATA: (8,0,0)
DATA: (11,0,0)
DATA: (8,0,0)
DATA: (9,0,0)
DATA: (1,1,0)
DATA: (15,0,0)
DATA: (11,0,0)
DATA: (14,0,0)
DATA: (11,0,0)
DATA: (1,0,0)
DATA: (15,0,0)
DATA: (9,0,0)
DATA: (11,0,0)
DATA: (8,0,0)
DATA: (1,0,0)
DATA: (14,0,0)
DATA: (15,0,0)
DATA: (11,0,0)
DATA: (9,0,0)
DATA: (8,0,0)
DATA: (1,0,0)
DATA: (13,23,1)
DATA: (11,0,0)
DATA: (14,0,0)
DATA: (15,0,0)
DATA: (1,0,0)
DATA: (13,0,0)
DATA: (8,0,0)
DATA: (14,0,0)
DATA: (1,0,0)
DATA: (13,0,0)
DATA: (15,0,0)
DATA: (12,2,0)
DATA: (15,0,0)
DATA: (1,0,0)
DATA: (13,1,0)
DATA: (12,0,0)
DATA: (15,0,0)
DATA: (13,0,0)
DATA: (13,0,0)
DATA: (12,0,0)
DATA: (13,0,0)
DATA: (13,0,0)
DATA: (13,0,0)
DATA: (12,0,0)
DATA: (12,0,0)
DATA: (12,0,0)
DATA: (12,0,0)
DATA: (12,0,0)

YEAR: 22
FROG POPULATION: 34
FROG POPULATION: 34
DATA: (1,8,0)
DATA: (4,7,0)
DATA: (5,7,0)
DATA: (6,8,0)
DATA: (8,4,0)
DATA: (9,8,0)
DATA: (10,10,0)
DATA: (11,4,0)
DATA: (12,3,0)
DATA: (13,6,0)
DATA: (14,11,0)
DATA: (15,11,0)
DATA: (0,4,0)
DATA: (2,6,0)
DATA: (3,6,1)
FROG POPULATION: 34
DATA: (7,11,0)

YEAR: 23
FROG POPULATION: 34
DATA: (1,18,1)
DATA: (2,18,0)
DATA: (3,28,0)
DATA: (4,20,0)
DATA: (5,19,0)
DATA: (7,20,0)
DATA: (8,16,1)
DATA: (9,14,0)
DATA: (11,25,0)
DATA: (12,18,0)
DATA: (13,23,0)
DATA: (14,16,0)
DATA: (15,19,0)
DATA: (0,26,0)
FROG POPULATION: 34
DATA: (10,22,1)
DATA: (6,16,0)
FROG POPULATION: 34

YEAR: 24
FROG POPULATION: 34
DATA: (1,21,0)
DATA: (2,24,0)
DATA: (4,16,1)
DATA: (5,21,0)
DATA: (6,18,0)
DATA: (8,22,0)
DATA: (9,27,0)
DATA: (11,24,0)
DATA: (12,23,1)
DATA: (13,17,0)
DATA: (14,29,0)
DATA: (15,21,0)
DATA: (0,26,0)
FROG POPULATION: 34
DATA: (7,17,1)
FROG POPULATION: 34
DATA: (3,26,1)

YEAR: 25
FROG POPULATION: 34
DATA: (1,17,0)
DATA: (3,15,0)
DATA: (4,20,0)
DATA: (5,15,0)
DATA: (7,15,0)
DATA: (8,12,0)
DATA: (9,12,0)
DATA: (11,15,0)
DATA: (12,22,0)
DATA: (13,16,0)
DATA: (14,18,0)
DATA: (10,29,0)
DATA: (15,25,0)
DATA: (0,18,0)
FROG POPULATION: 34
DATA: (10,0,0)
DATA: (6,24,1)
FROG POPULATION: 34
DATA: (2,20,0)

YEAR: 26
FROG POPULATION: 34
DATA: (1,22,0)
DATA: (2,18,0)
DATA: (3,20,1)
DATA: (4,29,0)
DATA: (5,29,1)
DATA: (6,25,0)
DATA: (8,24,0)
DATA: (9,27,0)
DATA: (11,30,1)
DATA: (12,27,1)
DATA: (13,21,1)
DATA: (14,36,0)
DATA: (15,21,0)
DATA: (0,20,0)
FROG POPULATION: 34
DATA: (7,27,1)
DATA: (10,24,0)
FROG POPULATION: 34

YEAR: 27
FROG POPULATION: 34
DATA: (1,29,1)
DATA: (3,16,0)
DATA: (5,21,0)
DATA: (7,24,0)
DATA: (8,18,0)
DATA: (9,22,1)
DATA: (10,8,1)
DATA: (11,20,2)
DATA: (12,14,3)
DATA: (13,14,0)
DATA: (14,22,0)
DATA: (15,14,0)
DATA: (0,17,1)
DATA: (6,24,0)
FROG POPULATION: 34
DATA: (4,11,0)
FROG POPULATION: 34
DATA: (2,18,1)

YEAR: 28
FROG POPULATION: 34
DATA: (1,20,0)
DATA: (2,29,2)
DATA: (3,22,0)
DATA: (4,18,2)
DATA: (5,17,0)
DATA: (7,28,1)
DATA: (8,22,0)
DATA: (9,24,2)
DATA: (11,25,1)
DATA: (12,24,1)
DATA: (13,30,0)
DATA: (14,35,1)
DATA: (15,20,1)
DATA: (0,20,1)
FROG POPULATION: 34
DATA: (10,25,0)
FROG POPULATION: 34
DATA: (6,34,0)

YEAR: 29
FROG POPULATION: 34
DATA: (1,26,1)
DATA: (3,14,1)
DATA: (5,31,1)
DATA: (6,15,0)
DATA: (8,27,0)
DATA: (9,38,1)
DATA: (10,24,0)
DATA: (11,27,0)
DATA: (12,34,0)
DATA: (13,14,1)
DATA: (14,32,0)
DATA: (15,21,0)
DATA: (0,29,0)
FROG POPULATION: 34
DATA: (4,25,1)
FROG POPULATION: 34
DATA: (7,21,0)
DATA: (2,30,1)

YEAR: 30
FROG POPULATION: 34
DATA: (1,14,0)
DATA: (2,23,0)
DATA: (3,18,0)
DATA: (4,22,1)
DATA: (5,21,0)
DATA: (7,15,0)
DATA: (8,19,0)
DATA: (9,28,0)
DATA: (11,22,0)
DATA: (12,29,0)
DATA: (13,22,0)
DATA: (14,32,0)
DATA: (15,22,0)
DATA: (0,19,0)
DATA: (10,17,1)
FROG POPULATION: 34
FROG POPULATION: 34
DATA: (6,14,0)

YEAR: 31
FROG POPULATION: 34
DATA: (1,27,0)
DATA: (3,31,0)
DATA: (4,27,1)
DATA: (5,24,0)
DATA: (6,15,0)
DATA: (8,21,0)
DATA: (9,25,0)
DATA: (11,33,0)
DATA: (12,28,0)
DATA: (13,26,0)
DATA: (14,21,0)
DATA: (15,29,0)
DATA: (0,26,1)
FROG POPULATION: 34
DATA: (7,29,0)
FROG POPULATION: 34
DATA: (10,32,1)
DATA: (2,38,1)

YEAR: 32
FROG POPULATION: 34
DATA: (1,24,0)
DATA: (2,6,0)
DATA: (3,16,0)
DATA: (5,21,0)
DATA: (8,16,0)
DATA: (9,17,0)
DATA: (10,24,0)
DATA: (11,18,1)
DATA: (12,22,0)
DATA: (13,16,0)
DATA: (14,17,1)
DATA: (15,35,1)
DATA: (0,14,0)
DATA: (6,18,1)
FROG POPULATION: 34
DATA: (7,25,0)
FROG POPULATION: 34
DATA: (4,25,1)

YEAR: 33
FROG POPULATION: 34
DATA: (1,31,1)
DATA: (3,28,1)
DATA: (4,12,2)
DATA: (5,31,2)
DATA: (7,21,1)
DATA: (8,31,1)
DATA: (9,28,1)
DATA: (11,25,1)
DATA: (12,22,0)
DATA: (13,27,1)
DATA: (14,22,1)
DATA: (15,22,1)
DATA: (0,17,0)
FROG POPULATION: 34
DATA: (6,33,0)
FROG POPULATION: 34
DATA: (10,33,1)
DATA: (2,36,0)

YEAR: 34
FROG POPULATION: 34
DATA: (1,25,1)
DATA: (2,15,0)
DATA: (3,20,0)
DATA: (5,18,0)
DATA: (6,29,1)
DATA: (8,24,0)
DATA: (9,34,0)
DATA: (10,17,0)
DATA: (11,26,0)
DATA: (12,21,0)
DATA: (13,25,0)
DATA: (14,22,0)
DATA: (15,26,0)
DATA: (0,14,0)
FROG POPULATION: 34
DATA: (7,19,1)
DATA: (4,7,0)
FROG POPULATION: 34

YEAR: 35
FROG POPULATION: 34
DATA: (3,26,3)
DATA: (4,7,1)
DATA: (5,33,4)
DATA: (7,30,2)
DATA: (8,29,0)
DATA: (9,25,0)
DATA: (11,31,3)
DATA: (13,22,2)
DATA: (14,31,2)
DATA: (15,24,1)
DATA: (0,29,0)
FROG POPULATION: 34
DATA: (6,14,0)
DATA: (10,34,2)
FROG POPULATION: 34
DATA: (2,36,4)

YEAR: 36
FROG POPULATION: 34
DATA: (2,1,0)
DATA: (3,7,0)
DATA: (5,4,0)
DATA: (6,3,0)
DATA: (8,2,1)
DATA: (9,4,0)
DATA: (10,1,0)
DATA: (11,0,0)
DATA: (14,6,0)
DATA: (0,3,0)
FROG POPULATION: 34
DATA: (7,0,0)
DATA: (4,9,0)
FROG POPULATION: 34

YEAR: 37
FROG POPULATION: 34
DATA: (3,2,0)
DATA: (4,0,0)
DATA: (5,1,0)
DATA: (6,2,0)
DATA: (8,3,0)
DATA: (9,2,0)
DATA: (14,1,0)
DATA: (0,2,0)
FROG POPULATION: 34
DATA: (10,5,0)
FROG POPULATION: 34
DATA: (2,2,0)
DATA: (7,2,0)

YEAR: 38
FROG POPULATION: 34
DATA: (2,0,0)
DATA: (3,0,0)
DATA: (5,0,0)
DATA: (7,0,0)
DATA: (8,0,0)
DATA: (9,0,0)
DATA: (10,1,0)
DATA: (14,0,0)
DATA: (0,0,0)
FROG POPULATION: 34
DATA: (4,0,0)
FROG POPULATION: 34
DATA: (6,0,0)

YEAR: 39
FROG POPULATION: 34
DATA: (3,0,0)
DATA: (4,0,0)
DATA: (5,0,0)
DATA: (6,0,0)
DATA: (9,0,0)
DATA: (10,0,0)
DATA: (0,0,0)
FROG POPULATION: 34
FROG POPULATION: 34
DATA: (7,0,0)
DATA: (2,0,0)

YEAR: 40
FROG POPULATION: 34
DATA: (2,0,0)
DATA: (3,0,0)
DATA: (4,0,0)
DATA: (7,0,0)
DATA: (0,0,0)
FROG POPULATION: 34
FROG POPULATION: 34
DATA: (6,0,0)
DATA: (10,0,0)

YEAR: 41
FROG POPULATION: 34
DATA: (3,0,0)
DATA: (6,0,0)
DATA: (10,0,0)
DATA: (0,0,0)
FROG POPULATION: 34
DATA: (7,0,0)
FROG POPULATION: 34
DATA: (4,0,0)
DATA: (2,0,0)
DATA: (1,27,0)

YEAR: 42
FROG POPULATION: 34
FROG POPULATION: 34
DATA: (5,0,0)
DATA: (8,0,0)
DATA: (9,0,0)
DATA: (12,24,1)
DATA: (13,2,0)
DATA: (14,0,0)
DATA: (15,4,0)
DATA: (15,0,0)
DATA: (9,0,0)
DATA: (8,0,0)
DATA: (5,0,0)
DATA: (14,0,0)
DATA: (15,0,0)
DATA: (8,0,0)
DATA: (13,2,0)
DATA: (12,5,0)
DATA: (14,0,0)
DATA: (15,0,0)
DATA: (13,0,0)
DATA: (15,0,0)
DATA: (1,5,1)
DATA: (15,0,0)
DATA: (13,0,0)
DATA: (12,3,0)
DATA: (1,1,0)
DATA: (12,1,0)
DATA: (13,0,0)
DATA: (13,0,0)
DATA: (12,0,0)
DATA: (12,0,0)
DATA: (12,0,0)
DATA: (1,2,0)
DATA: (1,0,0)
DATA: (1,0,0)
DATA: (1,0,0)
DATA: (1,6,0)
DATA: (2,6,0)
DATA: (3,9,0)
DATA: (4,8,1)
DATA: (5,11,1)
DATA: (6,7,1)
DATA: (8,6,0)
DATA: (9,6,0)
DATA: (12,4,1)
DATA: (13,6,0)
DATA: (14,15,3)
DATA: (15,5,0)
DATA: (0,2,0)
FROG POPULATION: 34
DATA: (11,1,0)
DATA: (11,0,0)
DATA: (11,0,0)
DATA: (11,0,0)
DATA: (11,0,0)
DATA: (11,5,1)
DATA: (7,3,0)
DATA: (10,2,0)

YEAR: 43
FROG POPULATION: 34
FROG POPULATION: 34
DATA: (1,12,1)
DATA: (3,15,0)
DATA: (5,14,0)
DATA: (7,9,0)
DATA: (8,15,0)
DATA: (9,11,0)
DATA: (10,10,0)
DATA: (11,18,0)
DATA: (12,10,0)
DATA: (13,16,0)
DATA: (14,14,0)
DATA: (15,11,0)
DATA: (0,20,0)
FROG POPULATION: 34
DATA: (4,16,0)
DATA: (2,14,1)
DATA: (6,12,0)

YEAR: 44
FROG POPULATION: 34
FROG POPULATION: 34
DATA: (1,14,0)
DATA: (2,22,1)
DATA: (3,16,1)
DATA: (4,19,3)
DATA: (5,14,1)
DATA: (6,13,1)
DATA: (8,17,0)
DATA: (9,14,2)
DATA: (11,9,1)
DATA: (12,15,1)
DATA: (13,16,1)
DATA: (14,17,1)
DATA: (15,14,2)
DATA: (0,16,3)
FROG POPULATION: 34
DATA: (10,0,0)
DATA: (7,17,0)

YEAR: 45
FROG POPULATION: 34
DATA: (1,16,0)
DATA: (3,18,0)
DATA: (5,4,0)
DATA: (7,13,0)
DATA: (8,15,1)
DATA: (9,12,0)
DATA: (10,9,0)
DATA: (11,12,0)
DATA: (12,14,0)
DATA: (13,10,0)
DATA: (14,11,0)
DATA: (15,15,0)
DATA: (0,12,0)
FROG POPULATION: 34
DATA: (4,0,0)
DATA: (2,0,0)
DATA: (6,12,0)
FROG POPULATION: 34

YEAR: 46
FROG POPULATION: 34
DATA: (1,15,0)
DATA: (2,9,0)
DATA: (3,10,0)
DATA: (4,11,0)
DATA: (5,13,0)
DATA: (6,14,0)
DATA: (8,17,0)
DATA: (9,8,0)
DATA: (11,16,1)
DATA: (12,11,0)
DATA: (13,15,1)
DATA: (14,28,0)
DATA: (15,11,0)
DATA: (0,12,0)
FROG POPULATION: 34
DATA: (10,7,0)
DATA: (7,14,0)
FROG POPULATION: 34

YEAR: 47
FROG POPULATION: 34
DATA: (1,7,0)
DATA: (3,5,0)
DATA: (5,14,1)
DATA: (7,7,0)
DATA: (8,14,1)
DATA: (9,15,0)
DATA: (10,7,0)
DATA: (11,11,0)
DATA: (12,7,2)
DATA: (13,13,0)
DATA: (14,14,1)
DATA: (15,11,1)
DATA: (0,12,0)
FROG POPULATION: 34
DATA: (6,0,0)
DATA: (4,6,0)
DATA: (2,0,0)
FROG POPULATION: 34

YEAR: 48
FROG POPULATION: 34
DATA: (1,9,0)
DATA: (2,8,0)
DATA: (3,9,0)
DATA: (4,8,1)
DATA: (5,8,0)
DATA: (6,5,0)
DATA: (8,10,0)
DATA: (9,12,0)
DATA: (11,13,0)
DATA: (12,13,0)
DATA: (13,7,0)
DATA: (14,11,0)
DATA: (15,3,0)
DATA: (0,11,0)
FROG POPULATION: 34
DATA: (7,0,0)
DATA: (10,18,1)
FROG POPULATION: 34

YEAR: 49
FROG POPULATION: 34
DATA: (1,8,1)
DATA: (3,8,1)
DATA: (5,6,0)
DATA: (7,10,0)
DATA: (8,4,1)
DATA: (9,8,0)
DATA: (10,6,0)
DATA: (11,5,0)
DATA: (12,6,0)
DATA: (13,7,0)
DATA: (14,4,0)
DATA: (15,6,0)
DATA: (0,10,1)
FROG POPULATION: 34
DATA: (2,0,0)
DATA: (4,7,0)
DATA: (6,0,0)
FROG POPULATION: 34

YEAR: 50
FROG POPULATION: 34
DATA: (1,20,0)
DATA: (2,12,0)
DATA: (3,8,0)
DATA: (4,14,1)
DATA: (5,20,0)
DATA: (6,16,0)
DATA: (8,14,0)
DATA: (9,12,0)
DATA: (11,13,0)
DATA: (12,17,0)
DATA: (13,10,0)
DATA: (14,13,0)
DATA: (15,13,0)
DATA: (0,8,0)
FROG POPULATION: 34
DATA: (7,11,1)
DATA: (10,16,0)
FROG POPULATION: 34

YEAR: 51
FROG POPULATION: 34
DATA: (1,24,0)
DATA: (2,25,0)
DATA: (3,23,2)
DATA: (4,32,1)
DATA: (5,25,1)
DATA: (6,34,0)
DATA: (8,34,0)
DATA: (9,17,1)
DATA: (11,25,0)
DATA: (12,43,0)
DATA: (13,20,0)
DATA: (14,32,1)
DATA: (15,17,1)
DATA: (0,32,3)
FROG POPULATION: 34
FROG POPULATION: 34
DATA: (7,8,1)
DATA: (10,11,0)

YEAR: 52
FROG POPULATION: 34
DATA: (1,10,0)
DATA: (3,15,0)
DATA: (5,17,0)
DATA: (7,18,0)
DATA: (8,8,0)
DATA: (9,9,0)
DATA: (10,13,0)
DATA: (11,14,0)
DATA: (12,13,0)
DATA: (13,11,0)
DATA: (14,8,0)
DATA: (15,9,0)
DATA: (0,15,1)
FROG POPULATION: 34
DATA: (6,8,0)
FROG POPULATION: 34
DATA: (2,13,1)
DATA: (4,13,0)

YEAR: 53
FROG POPULATION: 34
DATA: (1,20,0)
DATA: (2,15,0)
DATA: (3,17,0)
DATA: (4,18,0)
DATA: (5,16,0)
DATA: (6,15,0)
DATA: (8,15,0)
DATA: (9,12,0)
DATA: (11,16,0)
DATA: (12,17,0)
DATA: (13,22,0)
DATA: (14,15,0)
DATA: (15,12,0)
DATA: (0,14,0)
DATA: (7,18,1)
FROG POPULATION: 34
FROG POPULATION: 34
DATA: (10,15,0)

YEAR: 54
FROG POPULATION: 34
DATA: (1,16,1)
DATA: (3,13,2)
DATA: (5,15,0)
DATA: (6,18,0)
DATA: (8,17,0)
DATA: (9,20,3)
DATA: (10,16,1)
DATA: (11,23,1)
DATA: (12,15,1)
DATA: (13,13,0)
DATA: (14,10,1)
DATA: (15,20,0)
DATA: (0,25,1)
FROG POPULATION: 34
DATA: (2,17,0)
DATA: (4,24,1)
FROG POPULATION: 34
DATA: (7,27,1)

YEAR: 55
FROG POPULATION: 34
DATA: (2,16,0)
DATA: (3,12,0)
DATA: (4,13,0)
DATA: (5,18,0)
DATA: (7,4,0)
DATA: (8,13,0)
DATA: (9,18,0)
DATA: (11,23,0)
DATA: (13,13,1)
DATA: (14,25,0)
DATA: (15,14,0)
DATA: (0,19,0)
FROG POPULATION: 34
DATA: (10,0,0)
FROG POPULATION: 34
DATA: (6,17,0)

YEAR: 56
FROG POPULATION: 34
DATA: (3,1,0)
DATA: (5,4,0)
DATA: (6,2,0)
DATA: (8,2,0)
DATA: (9,2,0)
DATA: (10,6,0)
DATA: (11,2,0)
DATA: (14,5,0)
DATA: (0,6,0)
FROG POPULATION: 34
DATA: (2,0,0)
DATA: (4,0,0)
FROG POPULATION: 34
DATA: (7,2,0)

YEAR: 57
FROG POPULATION: 34
DATA: (2,0,0)
DATA: (3,1,0)
DATA: (4,2,0)
DATA: (5,1,0)
DATA: (7,0,0)
DATA: (8,0,0)
DATA: (9,1,0)
DATA: (11,4,0)
DATA: (14,1,0)
DATA: (0,1,0)
FROG POPULATION: 34
DATA: (10,0,0)
DATA: (6,2,0)
FROG POPULATION: 34

YEAR: 58
FROG POPULATION: 34
DATA: (3,0,0)
DATA: (5,0,0)
DATA: (6,0,0)
DATA: (8,0,0)
DATA: (9,0,0)
DATA: (10,0,0)
DATA: (14,0,0)
DATA: (0,0,0)
FROG POPULATION: 34
DATA: (2,0,0)
DATA: (4,0,0)
DATA: (7,0,0)
FROG POPULATION: 34

YEAR: 59
FROG POPULATION: 34
DATA: (2,0,0)
DATA: (4,0,0)
DATA: (7,0,0)
DATA: (10,0,0)
DATA: (0,0,0)
FROG POPULATION: 34
DATA: (6,0,0)
FROG POPULATION: 34

YEAR: 60
FROG POPULATION: 34
DATA: (2,0,0)
DATA: (4,0,0)
DATA: (6,0,0)
DATA: (0,0,0)
FROG POPULATION: 34
DATA: (7,0,0)
FROG POPULATION: 34
DATA: (10,0,0)

YEAR: 61
FROG POPULATION: 34
DATA: (7,0,0)
DATA: (10,0,0)
DATA: (0,0,0)
FROG POPULATION: 34
DATA: (6,0,0)
FROG POPULATION: 34
DATA: (2,0,0)
DATA: (4,0,0)

YEAR: 62
FROG POPULATION: 34
FROG POPULATION: 34
DATA: (1,16,1)
DATA: (3,0,0)
DATA: (5,0,0)
DATA: (8,0,0)
DATA: (9,0,0)
DATA: (11,0,0)
DATA: (11,0,0)
DATA: (5,0,0)
DATA: (12,13,0)
DATA: (13,2,0)
DATA: (14,0,0)
DATA: (8,0,0)
DATA: (5,0,0)
DATA: (3,0,0)
DATA: (14,0,0)
DATA: (11,0,0)
DATA: (15,2,0)
DATA: (8,0,0)
DATA: (3,0,0)
DATA: (11,0,0)
DATA: (14,0,0)
DATA: (15,1,0)
DATA: (9,0,0)
DATA: (15,0,0)
DATA: (13,2,0)
DATA: (9,0,0)
DATA: (15,0,0)
DATA: (13,0,0)
DATA: (15,0,0)
DATA: (13,0,0)
DATA: (12,3,0)
DATA: (1,5,0)
DATA: (15,0,0)
DATA: (13,0,0)
DATA: (1,0,0)
DATA: (1,0,0)
DATA: (12,2,0)
DATA: (13,0,0)
DATA: (1,0,0)
DATA: (12,0,0)
DATA: (1,0,0)
DATA: (12,0,0)
DATA: (1,0,0)
DATA: (12,0,0)
DATA: (12,0,0)
DATA: (1,5,0)
DATA: (2,5,1)
DATA: (3,5,0)
DATA: (4,11,0)
DATA: (5,7,0)
DATA: (6,10,0)
DATA: (8,8,0)
DATA: (9,11,0)
DATA: (10,8,0)
DATA: (11,12,1)
DATA: (12,13,0)
DATA: (13,18,0)
DATA: (14,5,0)
DATA: (15,17,0)
DATA: (0,2,0)
FROG POPULATION: 34
DATA: (7,22,0)

YEAR: 63
FROG POPULATION: 34
FROG POPULATION: 34
DATA: (1,56,0)
DATA: (2,48,0)
DATA: (3,48,1)
DATA: (5,49,1)
DATA: (7,25,1)
DATA: (8,55,1)
DATA: (9,46,2)
DATA: (10,40,0)
DATA: (11,48,2)
DATA: (12,50,0)
DATA: (13,45,1)
DATA: (14,27,0)
DATA: (15,34,1)
DATA: (0,72,0)
DATA: (4,29,0)
FROG POPULATION: 34
DATA: (6,39,0)

YEAR: 64
FROG POPULATION: 34
DATA: (1,34,0)
DATA: (2,50,1)
DATA: (3,55,1)
DATA: (4,38,0)
DATA: (5,50,2)
DATA: (6,26,2)
DATA: (8,48,1)
DATA: (9,37,0)
DATA: (10,40,0)
DATA: (11,65,2)
DATA: (12,41,1)
DATA: (13,46,1)
DATA: (14,58,2)
DATA: (15,50,0)
DATA: (0,44,2)
FROG POPULATION: 34
FROG POPULATION: 34
DATA: (7,25,1)

YEAR: 65
FROG POPULATION: 34
DATA: (1,33,1)
DATA: (2,43,1)
DATA: (3,28,2)
DATA: (5,33,0)
DATA: (7,17,0)
DATA: (8,33,0)
DATA: (9,43,0)
DATA: (10,40,1)
DATA: (11,24,1)
DATA: (12,44,0)
DATA: (13,28,0)
DATA: (14,31,0)
DATA: (15,36,0)
DATA: (0,40,0)
FROG POPULATION: 34
DATA: (4,25,0)
FROG POPULATION: 34
DATA: (6,48,1)

YEAR: 66
FROG POPULATION: 34
DATA: (1,45,0)
DATA: (2,36,0)
DATA: (3,39,1)
DATA: (5,31,0)
DATA: (6,29,0)
DATA: (8,44,0)
DATA: (9,49,1)
DATA: (10,37,1)
DATA: (11,37,0)
DATA: (12,35,1)
DATA: (13,33,2)
DATA: (14,38,1)
DATA: (15,42,0)
DATA: (0,49,0)
FROG POPULATION: 34
FROG POPULATION: 34
DATA: (4,40,0)
DATA: (7,40,1)

YEAR: 67
FROG POPULATION: 34
DATA: (1,24,0)
DATA: (2,27,1)
DATA: (3,27,2)
DATA: (4,29,0)
DATA: (5,22,1)
DATA: (7,30,0)
DATA: (8,27,0)
DATA: (9,27,2)
DATA: (10,32,2)
DATA: (11,18,0)
DATA: (12,26,0)
DATA: (13,33,1)
DATA: (14,34,0)
DATA: (15,29,1)
DATA: (0,29,0)
FROG POPULATION: 34
FROG POPULATION: 34
DATA: (6,33,1)

YEAR: 68
FROG POPULATION: 34
DATA: (1,24,0)
DATA: (2,14,0)
DATA: (3,20,0)
DATA: (5,20,0)
DATA: (6,10,0)
DATA: (8,18,0)
DATA: (9,16,0)
DATA: (10,15,0)
DATA: (11,26,0)
DATA: (12,22,0)
DATA: (13,20,1)
DATA: (14,16,0)
DATA: (15,17,0)
DATA: (0,9,0)
FROG POPULATION: 34
FROG POPULATION: 34
DATA: (4,16,0)
DATA: (7,22,1)

YEAR: 69
FROG POPULATION: 34
DATA: (1,37,0)
DATA: (2,30,0)
DATA: (3,19,0)
DATA: (4,19,0)
DATA: (5,21,0)
DATA: (7,28,0)
DATA: (8,24,1)
DATA: (9,19,0)
DATA: (10,30,0)
DATA: (11,29,0)
DATA: (12,21,0)
DATA: (13,21,0)
DATA: (14,22,0)
DATA: (15,14,1)
DATA: (0,28,0)
FROG POPULATION: 34
FROG POPULATION: 34
DATA: (6,32,1)

YEAR: 70
FROG POPULATION: 34
DATA: (1,18,0)
DATA: (2,16,0)
DATA: (3,13,0)
DATA: (5,11,0)
DATA: (6,11,0)
DATA: (8,18,0)
DATA: (9,20,0)
DATA: (10,17,0)
DATA: (11,16,0)
DATA: (12,21,0)
DATA: (13,20,0)
DATA: (14,17,0)
DATA: (15,8,0)
DATA: (0,16,0)
FROG POPULATION: 34
DATA: (4,4,0)
DATA: (7,21,1)
FROG POPULATION: 34

YEAR: 71
FROG POPULATION: 34
DATA: (1,39,0)
DATA: (2,30,0)
DATA: (3,22,0)
DATA: (4,43,1)
DATA: (5,31,0)
DATA: (7,33,1)
DATA: (8,22,1)
DATA: (9,30,0)
DATA: (10,35,1)
DATA: (11,32,0)
DATA: (12,33,1)
DATA: (13,33,2)
DATA: (14,38,1)
DATA: (15,28,1)
DATA: (0,32,0)
FROG POPULATION: 34
DATA: (6,32,1)
FROG POPULATION: 34

YEAR: 72
FROG POPULATION: 34
DATA: (1,104,3)
DATA: (2,93,3)
DATA: (3,119,3)
DATA: (4,102,7)
DATA: (5,105,5)
DATA: (6,96,4)
DATA: (7,116,9)
DATA: (8,88,3)
DATA: (9,120,7)
DATA: (10,109,3)
DATA: (11,121,7)
DATA: (12,119,7)
DATA: (13,117,3)
DATA: (14,112,6)
DATA: (15,120,5)
DATA: (0,89,1)
FROG POPULATION: 34
FROG POPULATION: 34

YEAR: 73
FROG POPULATION: 34
DATA: (1,384,24)
DATA: (2,426,50)
DATA: (3,368,28)
DATA: (4,400,64)
DATA: (5,421,40)
DATA: (6,398,43)
DATA: (7,401,34)
DATA: (8,392,36)
DATA: (9,397,47)
DATA: (10,409,45)
DATA: (11,395,44)
DATA: (12,401,29)
DATA: (13,418,48)
DATA: (14,422,48)
DATA: (15,400,40)
DATA: (0,395,39)
FROG POPULATION: 34
FROG POPULATION: 34

YEAR: 74
FROG POPULATION: 33
DATA: (1,407,108)
DATA: (2,391,100)
DATA: (3,409,114)
DATA: (4,410,127)
DATA: (5,408,109)
DATA: (6,377,105)
DATA: (7,397,107)
DATA: (8,450,115)
DATA: (9,424,135)
DATA: (10,424,120)
DATA: (11,440,118)
DATA: (12,422,109)
DATA: (13,399,107)
DATA: (14,385,107)
DATA: (15,370,105)
DATA: (0,424,119)
FROG POPULATION: 33
FROG POPULATION: 33

YEAR: 75
FROG POPULATION: 34
DATA: (1,403,290)
DATA: (2,391,284)
DATA: (3,428,316)
DATA: (4,406,287)
DATA: (5,385,307)
DATA: (6,439,322)
DATA: (7,442,328)
DATA: (8,428,304)
DATA: (9,412,311)
DATA: (10,398,304)
DATA: (11,422,307)
DATA: (12,441,329)
DATA: (13,417,314)
DATA: (14,420,324)
DATA: (15,387,279)
DATA: (0,451,312)
FROG POPULATION: 34
FROG POPULATION: 33

YEAR: 76
FROG POPULATION: 33
DATA: (1,411,404)
DATA: (2,420,409)
DATA: (3,402,395)
DATA: (4,404,396)
DATA: (5,439,428)
DATA: (6,440,427)
DATA: (7,422,408)
DATA: (8,403,395)
DATA: (9,420,415)
DATA: (10,426,420)
DATA: (11,435,425)
DATA: (12,384,381)
DATA: (13,478,465)
DATA: (14,428,418)
DATA: (15,419,413)
DATA: (0,464,447)
FROG POPULATION: 33
FROG POPULATION: 34

YEAR: 77
FROG POPULATION: 32
DATA: (1,403,394)
DATA: (2,430,424)
DATA: (3,411,406)
DATA: (4,430,420)
DATA: (5,444,435)
DATA: (6,386,378)
DATA: (7,412,400)
DATA: (8,428,418)
DATA: (9,450,439)
DATA: (10,408,398)
DATA: (11,444,437)
DATA: (12,461,452)
DATA: (13,428,420)
DATA: (14,402,391)
DATA: (15,418,403)
DATA: (0,393,389)
FROG POPULATION: 32
FROG POPULATION: 32

YEAR: 78
FROG POPULATION: 33
DATA: (1,408,402)
DATA: (2,383,372)
DATA: (3,356,346)
DATA: (4,359,351)
DATA: (5,409,400)
DATA: (6,390,384)
DATA: (7,353,347)
DATA: (8,387,379)
DATA: (9,390,382)
DATA: (10,369,361)
DATA: (11,385,375)
DATA: (12,373,361)
DATA: (13,381,372)
DATA: (14,401,395)
DATA: (15,406,399)
DATA: (0,378,374)
FROG POPULATION: 32
FROG POPULATION: 31

YEAR: 79
FROG POPULATION: 31
DATA: (1,330,315)
DATA: (2,341,329)
DATA: (3,356,335)
DATA: (4,351,340)
DATA: (5,352,333)
DATA: (6,345,326)
DATA: (7,352,336)
DATA: (8,338,317)
DATA: (9,348,332)
DATA: (10,376,360)
DATA: (11,354,340)
DATA: (12,331,316)
DATA: (13,342,328)
DATA: (14,339,326)
DATA: (15,361,345)
DATA: (0,323,307)
FROG POPULATION: 31
FROG POPULATION: 32

YEAR: 80
FROG POPULATION: 33
DATA: (1,325,306)
DATA: (2,325,310)
DATA: (3,310,295)
DATA: (4,327,303)
DATA: (5,267,254)
DATA: (6,317,295)
DATA: (7,313,302)
DATA: (8,293,275)
DATA: (9,291,276)
DATA: (10,297,285)
DATA: (11,328,305)
DATA: (12,331,316)
DATA: (13,304,296)
DATA: (14,307,293)
DATA: (15,289,277)
DATA: (0,288,266)
FROG POPULATION: 33
FROG POPULATION: 33

YEAR: 81
FROG POPULATION: 31
DATA: (1,305,292)
DATA: (2,278,273)
DATA: (3,316,308)
DATA: (4,300,296)
DATA: (5,305,294)
DATA: (6,292,280)
DATA: (7,288,279)
DATA: (8,289,282)
DATA: (9,286,284)
DATA: (10,278,271)
DATA: (11,265,258)
DATA: (12,290,281)
DATA: (13,283,276)
DATA: (14,320,314)
DATA: (15,307,298)
DATA: (0,307,298)
FROG POPULATION: 30
FROG POPULATION: 30

YEAR: 82
FROG POPULATION: 30
DATA: (3,84,84)
DATA: (4,87,87)
DATA: (6,76,75)
DATA: (7,82,80)
DATA: (8,93,90)
DATA: (9,70,68)
DATA: (10,98,95)
DATA: (11,92,91)
DATA: (12,77,76)
DATA: (13,83,81)
DATA: (14,80,80)
DATA: (15,87,86)
DATA: (0,104,101)
FROG POPULATION: 30
FROG POPULATION: 30

YEAR: 83
FROG POPULATION: 30
DATA: (3,0,0)
DATA: (4,0,0)
DATA: (6,0,0)
DATA: (7,0,0)
DATA: (8,0,0)
DATA: (9,0,0)
DATA: (10,0,0)
DATA: (11,0,0)
DATA: (14,0,0)
DATA: (0,0,0)
FROG POPULATION: 30
FROG POPULATION: 30

YEAR: 84
FROG POPULATION: 30
DATA: (3,0,0)
DATA: (4,0,0)
DATA: (6,0,0)
DATA: (7,0,0)
DATA: (8,0,0)
DATA: (10,0,0)
DATA: (14,0,0)
DATA: (0,0,0)
FROG POPULATION: 30
FROG POPULATION: 30

YEAR: 85
FROG POPULATION: 30
DATA: (3,0,0)
DATA: (4,0,0)
DATA: (6,0,0)
DATA: (7,0,0)
DATA: (8,0,0)
DATA: (10,0,0)
DATA: (14,0,0)
DATA: (0,0,0)
FROG POPULATION: 30
FROG POPULATION: 30

YEAR: 86
FROG POPULATION: 30
DATA: (3,0,0)
DATA: (4,0,0)
DATA: (6,0,0)
DATA: (7,0,0)
DATA: (8,0,0)
DATA: (10,0,0)
DATA: (14,0,0)
DATA: (0,0,0)
FROG POPULATION: 30
FROG POPULATION: 30

YEAR: 87
FROG POPULATION: 30
DATA: (4,0,0)
DATA: (6,0,0)
DATA: (7,0,0)
DATA: (0,0,0)
FROG POPULATION: 30
FROG POPULATION: 30

YEAR: 88
FROG POPULATION: 30
DATA: (4,0,0)
DATA: (6,0,0)
DATA: (7,0,0)
DATA: (0,0,0)
FROG POPULATION: 30
FROG POPULATION: 30

YEAR: 89
FROG POPULATION: 30
DATA: (6,0,0)
DATA: (0,0,0)
FROG POPULATION: 30
DATA: (12,0,0)
DATA: (13,0,0)
DATA: (14,0,0)
DATA: (15,0,0)
DATA: (1,109,107)
DATA: (2,82,81)
DATA: (3,0,0)
DATA: (4,0,0)
DATA: (7,0,0)
DATA: (8,0,0)
DATA: (9,0,0)
DATA: (10,0,0)
DATA: (11,0,0)
DATA: (1,0,0)
DATA: (11,0,0)
DATA: (12,0,0)
DATA: (13,0,0)
DATA: (2,0,0)
DATA: (15,0,0)
DATA: (8,0,0)
DATA: (1,0,0)
DATA: (13,0,0)
DATA: (12,0,0)
DATA: (11,0,0)
DATA: (3,0,0)
DATA: (14,0,0)
DATA: (2,0,0)
DATA: (9,0,0)
DATA: (1,0,0)
DATA: (13,0,0)
DATA: (15,0,0)
DATA: (11,0,0)
DATA: (2,0,0)
DATA: (12,0,0)
DATA: (14,0,0)
DATA: (1,0,0)
DATA: (13,0,0)
DATA: (8,0,0)
DATA: (15,0,0)
DATA: (10,0,0)
DATA: (3,0,0)
DATA: (11,0,0)
DATA: (2,0,0)
DATA: (9,0,0)
DATA: (12,0,0)
DATA: (1,0,0)
DATA: (13,0,0)
DATA: (15,0,0)
DATA: (2,0,0)
DATA: (11,0,0)
DATA: (10,0,0)
DATA: (12,0,0)
DATA: (9,0,0)
DATA: (1,0,0)
DATA: (13,0,0)
DATA: (2,0,0)
DATA: (15,0,0)
DATA: (12,0,0)
DATA: (9,0,0)
DATA: (1,0,0)
DATA: (15,0,0)
DATA: (2,0,0)
DATA: (9,0,0)

YEAR: 90
FROG POPULATION: 31
FROG POPULATION: 31
DATA: (5,84,83)
DATA: (5,0,0)
DATA: (5,0,0)
DATA: (5,0,0)
DATA: (5,0,0)
DATA: (5,0,0)
DATA: (5,0,0)
DATA: (5,0,0)
DATA: (1,24,20)
DATA: (2,24,21)
DATA: (3,23,23)
DATA: (4,41,41)
DATA: (5,31,28)
DATA: (6,27,26)
DATA: (7,27,23)
DATA: (8,31,30)
DATA: (9,43,40)
DATA: (10,31,30)
DATA: (11,42,37)
DATA: (12,31,30)
DATA: (13,36,34)
DATA: (14,43,39)
DATA: (15,39,36)
DATA: (0,19,18)
FROG POPULATION: 31

YEAR: 91
FROG POPULATION: 30
FROG POPULATION: 30
DATA: (1,363,338)
DATA: (2,401,380)
DATA: (3,363,343)
DATA: (4,393,375)
DATA: (5,367,348)
DATA: (6,362,339)
DATA: (7,373,358)
DATA: (8,366,345)
DATA: (9,419,394)
DATA: (10,380,358)
DATA: (11,391,363)
DATA: (12,416,396)
DATA: (13,423,399)
DATA: (14,359,341)
DATA: (15,404,380)
DATA: (0,388,372)
FROG POPULATION: 31

YEAR: 92
FROG POPULATION: 29
DATA: (1,353,345)
DATA: (2,365,348)
DATA: (3,335,324)
DATA: (4,332,322)
DATA: (5,346,330)
DATA: (6,374,360)
DATA: (7,369,357)
DATA: (8,352,337)
DATA: (9,371,354)
DATA: (10,362,345)
DATA: (11,398,381)
DATA: (12,388,370)
DATA: (13,387,373)
DATA: (14,391,379)
DATA: (15,406,393)
DATA: (0,367,354)
FROG POPULATION: 29
FROG POPULATION: 27

YEAR: 93
FROG POPULATION: 27
DATA: (1,333,333)
DATA: (2,315,315)
DATA: (3,326,326)
DATA: (4,378,378)
DATA: (5,302,302)
DATA: (6,324,324)
DATA: (7,317,317)
DATA: (8,322,322)
DATA: (9,385,385)
DATA: (10,330,330)
DATA: (12,341,341)
DATA: (11,349,349)
DATA: (13,327,327)
DATA: (14,335,335)
DATA: (15,326,326)
DATA: (0,356,356)
FROG POPULATION: 27
FROG POPULATION: 26

YEAR: 94
FROG POPULATION: 27
DATA: (1,337,335)
DATA: (2,341,340)
DATA: (3,356,354)
DATA: (4,341,340)
DATA: (5,322,320)
DATA: (6,323,319)
DATA: (7,327,323)
DATA: (8,337,337)
DATA: (9,316,315)
DATA: (10,330,329)
DATA: (11,361,359)
DATA: (12,356,355)
DATA: (13,334,333)
DATA: (14,295,294)
DATA: (15,345,344)
DATA: (0,358,355)
FROG POPULATION: 27
FROG POPULATION: 26

YEAR: 95
FROG POPULATION: 26
DATA: (1,336,329)
DATA: (2,381,372)
DATA: (3,340,330)
DATA: (4,330,320)
DATA: (5,355,347)
DATA: (6,343,332)
DATA: (7,333,325)
DATA: (8,348,342)
DATA: (10,357,352)
DATA: (9,359,345)
DATA: (11,354,344)
DATA: (12,335,323)
DATA: (13,335,325)
DATA: (14,357,346)
DATA: (15,367,359)
DATA: (0,322,313)
FROG POPULATION: 26
FROG POPULATION: 25

YEAR: 96
FROG POPULATION: 24
DATA: (1,333,329)
DATA: (2,360,359)
DATA: (3,354,351)
DATA: (4,321,320)
DATA: (5,341,336)
DATA: (6,367,361)
DATA: (7,332,330)
DATA: (8,345,343)
DATA: (9,328,324)
DATA: (10,346,343)
DATA: (11,371,367)
DATA: (12,351,349)
DATA: (13,314,311)
DATA: (14,321,320)
DATA: (15,348,345)
DATA: (0,319,315)
FROG POPULATION: 24
FROG POPULATION: 24

YEAR: 97
FROG POPULATION: 24
DATA: (1,327,327)
DATA: (2,308,308)
DATA: (3,345,345)
DATA: (4,300,300)
DATA: (5,351,351)
DATA: (6,345,345)
DATA: (7,278,278)
DATA: (8,326,326)
DATA: (9,341,341)
DATA: (10,351,351)
DATA: (11,289,289)
DATA: (12,311,311)
DATA: (13,352,352)
DATA: (14,349,349)
DATA: (15,323,323)
DATA: (0,324,324)
FROG POPULATION: 24
FROG POPULATION: 24

YEAR: 98
FROG POPULATION: 23
DATA: (1,293,288)
DATA: (2,329,323)
DATA: (3,322,319)
DATA: (4,321,316)
DATA: (5,302,302)
DATA: (6,291,288)
DATA: (7,321,320)
DATA: (8,321,319)
DATA: (9,307,303)
DATA: (10,292,289)
DATA: (11,316,310)
DATA: (13,306,301)
DATA: (12,295,293)
DATA: (14,298,294)
DATA: (15,344,343)
DATA: (0,306,300)
FROG POPULATION: 22
FROG POPULATION: 20

YEAR: 99
FROG POPULATION: 20
DATA: (1,313,313)
DATA: (2,295,295)
DATA: (3,276,276)
DATA: (4,260,260)
DATA: (5,265,265)
DATA: (6,273,273)
DATA: (7,298,298)
DATA: (8,290,290)
DATA: (10,272,272)
DATA: (11,282,282)
DATA: (12,286,286)
DATA: (9,287,287)
DATA: (13,260,260)
DATA: (14,326,326)
DATA: (15,309,309)
DATA: (0,294,294)
FROG POPULATION: 20
FROG POPULATION: 20

YEAR: 100
FROG POPULATION: 20
DATA: (1,255,255)
DATA: (2,251,251)
DATA: (3,282,282)
DATA: (4,323,323)
DATA: (5,278,278)
DATA: (6,282,282)
DATA: (7,251,251)
DATA: (8,295,295)
DATA: (10,299,299)
DATA: (9,286,286)
DATA: (11,254,254)
DATA: (12,283,283)
DATA: (13,292,292)
DATA: (14,319,319)
DATA: (15,275,275)
DATA: (0,271,271)
FROG POPULATION: 20
FROG POPULATION: 19
}

\end{document}
